\section{Conclusion}
\label{sec:conclusion}

In the previous chapters a simulation of autonomous vehicles in a multi-modal
traffic network has been developed and assessed. The development started out
with the extension of the existing Sioux Falls traffic network for MATSim to
a more finegrained resolution, with comparable traffic characteristics as the
original scenario.

Furthermore, a the new AgentLock framework for the simulation of dynamic agents in MATSim has
been developed, providing a decreased computational time for a wide range of
simulation configurations, compared to the existing DVRP extension of MATSim.
Additionally, a layer for the easy programming of state-based agents, AgentFSM has
been created and used in the subsequent simulations.

For the simulation of AVs, a basic agent logic has been shown, a dispatching
strategy has been assessed and different ways of routing AVs through the network
have been analyzed.

Finally, behavioral parameters for the simulation of AVs have been defined and
quantitative results of the AV simulation have been
demonstrated in detail. However, the simulation at hand is by nature more suited
for giving qualitative insights into the traffic system rather then quantitative
since it is based on a virtual test scenario.

One of the first results, which could be obtained, is that AVs in the test
scenario will lead to an increased overall milage, which is an adverse effect
looking from an environmentally perspective, but is also disadvantageous in
terms of congestion in the network. Implementation-wise it has been found that
considerable work needs to be put into intelligent ways of making AVs facilitate
the existing infrastructure in order to avoid artificially made traffic jams,
which is a problem that gets crucial with an increasing number of AV trips
and therefore increasing mileage.

In that regard the introduction of AVs plays against one positive effect of
public transport: to have fewer vehicles one the road. As could be shown in the
results, former public transport users are the main adapters of autonomous vehicles
in the given scenario. While private car users have a financial motivation to switch
to the AV mode and accept longer travel times, public transport users mainly do
the switch to reduce their accumulated travel time consisting of the walking
to the stop facilities and the ride itself, possibly with several line switches.

This choice behavior goes along with a high tolerance for waiting times for the
AV mode. Because public transport users are used to having a long travel time,
the performance that could be reached in the given network was clearly enough
to serve the demand. Nevertheless, the public transport users are not the
audience that a policy maker would want to attract with AVs. From the findings
in the simulations one can state (at least for the Sioux network), that it is
very hard to impossible to maintain a certain level of public transport while
getting private car owners to using the AV service. In fact, applying pinpointed
incentive or taxation schemes might be necessary to reach at the
desired results.

In terms of pricing it has been found that the more inclined people are to spend
time in an AV, the less constrained the financial structure of the service needs
to be in order to reach certain AV shares. For the baseline scenario it has been
found that it is possible to operate an AV service in a profitable way while still
maintaining a share of \%15 public transport and providing waiting times of less
than ten minutes. While doing this, there would be a financial margin available
on the side of the operator to cover necessary infrastructural expenses.

The final conclusion is that AVs without administrative regulation are likely to
attract public transport users rather than private car owners. Further research
needs to be done on how AV usage can be incentivized for private car owners to
reach at a beneficial traffic situation.

Finally, it remains to state that the proposed model is built on a manifold of
assumptions in the initial Sioux Falls scenario, in the operation of the AV
agents and on the financial model for the operator. While for the latter future
predictions will become better and better, it would be highly interesting to apply
the developed model to a real-world scenario with a higher confidence in the estimated
utility parameters and relative factors such as current taxi pricing.

\section{Outlook}
\label{sec:outlook}

During the course of this thesis a versatile and extentable basis model for the
simulation of autonomous vehicles has been developed. Many topics in the field
which would be interesting to investigate were not part of the scope of this thesis.
The following sections will present the most important and interesting ones and
give directions on how to simulate them with the framework.

\subsection{Infrastructure Extensions}

The adaptation of autonomous vehicles will take great advantage of the ongoing electrification in
the automotive industry. While the ``distance anxiety'', which makes people refrain
from using electric vehicles due to the limited driving range, is a major barrier
for the introduction of electric vehicles, autonomous vehicles might be able to
solve this problem by providing reliable trips through an operator \citep{Burmeister2016, Chen16}. Therefore
the availability of the respective electrification infrastructure is one factor which has to be taken
into account when predicting the adaption of autonomous vehicles.

Given the right amount of resources to create such an infrastructure, a big question
that arises is how many recharging facilities are needed and where they should begin
located in order to serve certain sizes of AV fleets \citep{Chen2015}. Furthermore, how can a combined
infrastructure for ordinary owned EVs, private AVs and publicly provided AV services
be designed?

As a first step one could assume that AVs recharge at dropoff locations, where they
are artificially put to rest for a certain minimum idle time in order to simulate
the recharging process. This would already lead to results on the customer acceptance
side, but would ignore effects on congestion if AVs would actually need to take long
trips to the next charging facility. Previous studies using MATSim already gave interesting results on the implementation
of eletrified taxis in Berlin with distributed charging facilities
over the network \citep{Bischoff2014}. Such research could be combined with the AV framework developed here.
The modular AgentFSM component would make it easy to add specific new points in the
state chains of an AV to drive to a charging facility. Then again, the simulation framework could be used to get insights in which scheduling
strategies would be optimal for an AV fleet if recharging has to be taken into account.

Another idea to improve the simulation is to introduce means of simulating maintenance
and parking. So far it has been assumed that autonomous vehicles will reside where the last
dropoff has taken place. This assumption is
quite optimistic, since parking space might be rare. In consequence, the driven
kilometers per AV in the unoccupied state could be quite off the actual distance.
More mileage would be needed to find a parking space in between tasks.
This rises a whole range of questions on how an optimal AV scheduling would
look like, maybe depending on the demand level, it might be even interesting to
run  studies on whether the search for parking space might be inferior
to roaming around.

This could be done in combination with intelligent repositioning, where in between peak
hours taxis could be intelligently moved to likely pickup positions and thus minimize the waiting
time for customers, increasing the acceptance and reducing
operator costs at the same time because less unoccupied miles might be travelled.

With the parallelization of customer trips the presented framework already shows
by example how such an algorithm could be incorporated into the existing infrastructure
without having too much impact on the computation times. In general, doing ``some''
intelligent repositioning should always be more beneficial than doing none. In
this regard, the repositioning could be computed in parallel to the ongoing
traffic simulation, while still offering the ability to restrict it if it slows
down the main loop of MATSim.

\subsection{Usage and interaction with the AV service}

The AV service in the developed model so far is very basic in the way customers
are able to interact with it. One obvious advantage of AV fleets is, that up to five people could be transported
in ordinarily shaped cars and even more in autonomous minibusses or full-sized busses.
Intelligent routing and scheduling algorithms would make it possible to pick up passengers
at arbitrary locations, not being bound to a fixed public transport schedule.

Due to the extensible structrue of the AV extension, its would be easy to
add such behaviour in principle. However, the problem of optimizing the trips of
more than one passenger, probably while already on a ride, is a highly complex
problem and heuristics are still an important research subject.
Adding such functionality to the AV framework would make it possible to test such
strategies in near-realistic scenarios and measure the performance of different
heuristics.

Additionally, autonomous vehicles are likely to relax the last mile problem, where people are not
motivated to opt for public transport, because the last mile from the transport
facility to their home is too long. An AV could bridge this distance, maybe being
prescheduled to pick up the passenger according to the current expected arrival
times.

Implementing this behaviour would need only small changes in the MATSim framework.
Generally, a public transport trip is generated as three legs: One \texttt{transit\_walk}
leg in order to get to the stop facility, one \texttt{pt} leg for the actual residual
time in the bus  and another \texttt{transit\_walk}. A first attempt
would be to replace some of the \texttt{transit\_walk} legs during the planning
phase with \texttt{av} trips, which could already lead to a convincing simulation
of AVs feeding the public transport network.
A probable requirement for this to work well would be to have prescheduled AVs
to give a higher reliability on these connections.

Prescheduled autonomous vehicles would be easy to implement in the existing infrastructure.
In fact, test have already been done, but abandoned due to the fact that in a first
approximation delays from the scheduling can be modeled using worse utility values
for the waiting times.
In the current state, the AV framework fully supports such prescheduled trips, though
their impact has not been investigated in the scope of this thesis. As shown in
figure \cref{fig:avstates}, where the state diagram of the AV is depicted, a ``Waiting''
state is already included, which would make an AV which arrives early at a pickup
location wait for the passenger.

An intereting point in the simulation is the spatial dependence. On one side,
it would be interesting to investigate where AV users live, maybe indicating that
on specific pricing strategies, people from the suburbs prefer AVs, while other
strategies might encourage people living in the center to use them.

Heavily related to that is the initial distribution of AVs at the beginning
of the daily simulation. As described before, AVs are currently distributed dependent on
the population density, though that might not be the best approach. Especially
if one wants to encourage suburbians to use AVs, the density there should be higher.

In that regard it would be beneficial to investigate how the initial conditions
in the MATSim simulation influence the result on an abstract level. For the case of
AVs, having initial AV plans mainly assigned to agents from a city center, but not for people
from the suburbs, the relaxed state might settle down in exactly this condition.
However, if AVs are mainly distributed in the suburbs in the initial plans, the
main user group might stay there, just because during the simulations the waiting
times are shorter in either case and therefore these people might stick to their
initial plan decisions. Therefore, a thorough investigation of the distribution
behavior of the algorithm would be very interesting.

Furthermore, the research in this thesis has shown that without any incentives, AVs might lead
to adverse effects, which should be corrected by intelligent policy decisions. \Citet{Chen16} suggests
interesting approaches of incentivizing AV usage. In order to ``free'' the city center
from too much congestion one could for instance in the Sioux Falls scenario put high monetary fees on using the streets within the
highway belt for private cars, while in parallel increasing the likelihood for AVs
to use the highways. This way one could try to move traffic to the highways, then
taking a direct trip to the workplaces in a perpendicular way.

Experiments like these are ready to be done with the existing simulation framework
by adding custom scoring functions for private cars and autonomous vehicles.

Finally, efforts have been made recently to diversify the population in MATSim simulations \citep{Chakirov2015}.
This means that people might be constrained or inclined due to age or income to use certain means
of transport. Combining the simulation of AVs with the introduction of that heterogeneity
could give insights on how the availability and pricing of AVs affect the distribution
of users in terms of a richer set of social variables.
