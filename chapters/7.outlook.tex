\section{Outlook}
\label{sec:outlook}

During the course of this thesis a versatile and extentable basis model for the
simulation of autonomous vehicles has been developed. Many topics in this field,
which would be interesting to investigate, were not part of the scope of this thesis.
The following sections will present the most important and interesting ones and
give directions on how to simulate them with the framework. Likewise, these sections
also give an overview about critical points, which are not taken into account in
the previous results, but could change the overall picture significantly.

\subsection{Recharging Infrastructure}

Since autonomous vehicles in general will take great advantage of the ongoing
electrification in the automotive industry, the availability of the respective
infrastructure is one factor, which can determine how fast the adaptation of AVs
will progress \todo{cite sth}.

Given the right amount of resources to create such an infrastructure, a big question
that arises is how many recharging facilities are needed and where they should begin
located in order to serve certain sizes of AV fleets \citep{Chen2015}. Furthermore, how can a combined
infrastructure for ordinary owned EVs, private AVs and publicly provided AV services
be designed?

As a first step one could assume that AVs recharge at dropoff locations, where they
are artificially put to rest for a certain minimum idle time in order to simulate
the recharging process. This would already lead to results on the customer acceptance
side, but would ignore effects on congestion if AVs would actually need to take long
trips to the next charging facility.

Previous studies using MATSim already gave interesting results on the implementation
of private EVs \todo{cite EVs Berlin} in Berlin with distributed charging facilities
over the network. Such research could be combined with the AV framework developed here.
The modular AgentFSM component would it make easy to add specific new points in the
state chains of an AV to drive to a charging facility.

Then again, the simulation framework could be used to get insights in which scheduling
strategies would be optimal for an AV fleet if recharging has to be taken into account.

\subsection{Shared Trips}

One obvious advantage of AV fleets is, that up to five people could be transported
in ordinarily shaped cars and even more in autonomous minibusses or full-sized busses.
Intelligent routing and scheduling algorithms would make it possible to pick up passengers
at arbitrary locations, not being bound to a fixed public transport schedule.

Again, due to the extensible structrue of the AV extension, it would be easy to
add such behaviour in principle. However, the problem of optimizing the trips of
more than one passenger, probably while already on a ride, is a highly complex
problem and acceptable heuristics are still an important research subject.

Adding such functionality to the AV framework would make it possible to test such
strategies in near-realistic scenarios and measure the performance of different
heuristics.

\todo{some citations here}

\subsection{The Last Mile Problem}

Autonomous vehicles are likely to relax the last mile problem, where people are not
motivated to opt for public transport, because the last mile from the transport
facility to their home is too long. An AV could bridge this distance, maybe being
prescheduled to pick up the passenger according on the current expected arrival
times.

Implementing this behaviour would need only small changes in the MATSim framework.
Generally, a public transport trip is generated as three legs: One \texttt{transit\_walk}
leg in order to get to the stop facility, one \texttt{pt} leg for the actual residual
time in the bus or train and another \texttt{transit\_walk}. A very easy first attempt
would be to replace some of the \texttt{transit\_walk} legs during the planning
phase with \texttt{av} trips and a quite convincing simulation of AVs feeding
public transport lines would be set up.

A probable requirement for this to work well would be to have prescheduled AVs,
which are more or less likely to arrive at a prescheduled time.

\subsection{Prescheduled AVs}

Prescheduled autonomous vehicles would be easy to implement in the existing infrastructure.
In fact, test have already be done, but abandoned due to the fact that in a first
approximation delays from the scheduling can be modeled using worse utility values
for the waiting times. At least this would lead to comparable results on the customer
acceptance side.

In the current state, the AV framework fully supports such prescheduled trips, though
their impact has not been investigated in the scope of this thesis. As shown in
figure \todo{reference}, where the state diagram of the AV was depicted, a ``Waiting''
state is already included, which would make an AV, which arrives early at a pickup
location, wait for the passenger.

\subsection{Intelligent Repositioning}

A big field of research on taxi services in general is how to reposition available
taxis while they are not in service. This means that before peak hours taxis could
be intelligently moved to likely pickup positions and thus minimize the waiting
time for customers, thus increasing the acceptance while at the same time reducing
operator costs because less unoccupied miles might be travelled (if the algorithm
is effective).

With the parallelization of customer trips the presented framework already shows
by example how such an algorithm could be incorporated into the existing infrastructure
without having too much impact on the computation times. In general, doing ``some''
intelligent repositioning should always be more beneficial than doing none. In
this regard, the repositioning could be computed in parallel to the ongoing
traffic simulation, while still offering the ability to restrict it if it slows
down the general computation of MATSim.

Alternatively, repositioning could of course be run serially and thus arriving
at complete heuristic or optimized solutions.

\subsection{Parking}

So far it has been assumed that autonomous vehicles will reside where the last
dropoff has taken place. This assumption, especially in heavily packed cities, is
quite optimistic, since parking space might be rare. In consequence, the driven
kilometers per AV in the unoccupied state could be quite off the actual distance
that would be needed to get to the next pickup location, but also to find a parking
space in between.

Again, this rises a whole range of question on how an optimal AV scheduling would
look like, maybe depending on the demand level, it might be even interesting to
run preliminary studies on whether the search for parking space might be inferior
to roaming (and this towards intelligently chosen locations).

\subsection{Heterogeneity}

Recently, efforts have been made to diversify the population in MATSim simulations.
This means that people might be constrained due to age or income to use certain means
of transport. Combining the simulation of AVs with the introduction on heterogeneity
(as previously done for the Sioux Falls scenario) could give insights on how the
availability and pricing of AVs affects the distribution of acceptance among diverse
social groups in a city.

\subsection{Adaptive Pricing Strategies}

Related to this are adaptive pricing strategies. Once it is established, which disadvantages
might be introduced by AVs (such as less accessibility to public transport for some social or
spatially remotely located groups), research can be done on how to even out those
differences by applying pinpointed pricing strategies. \Citep{Chen16} suggests
\todo{there are some interesting options, just name them here}

In order to ``free'' the city center from too much congestion one could for instance
in the Sioux Falls scenario put high monetary fees on using the streets within the
highway belt for private cars, while in parallel increasing the likelihood for AVs
to use the highways. This way one could try to move traffic to the highways, then
taking a direct trip to the workplaces in a perpendicular way.

Experiments like this are ready to be done with the existing simulation framework
by adding custom scoring functions for private cars and autonomous vehicles.

\subsection{Spatial Dependence}

Another intereting point is the spatial dependence in the simulation. On one side,
it would be interesting to investigate where AV users live, maybe indicating that
on specific pricing strategies, people from the suburbs prefer cars, while other
strategies might encourage people living in the center to use them.

An important point related to this is the initial distribution of AVs at the beginning
of the daily simulation. As described above, here AVs are distributed dependent on
the population density, though that might not be the best approach. Especially
if one wants to encourage suburbians to use AVs, the density there should be higher.

In that regard it would also be interesting to investigate how the initial conditions
in the MATSim simulation influence the result on an abstract level. For the case of
AVs, having initial plans to use AVs for a lot of center people, but not for people
from the suburbs, the relaxed state might settle down in exactly this condition.
However, if AVs are mainly distributed in the suburbs in the initial plans, the
main user group might stay there, just because during the simulations the waiting
times are shorter in either case and therefore these people might stick to their
initial plan decisions.
