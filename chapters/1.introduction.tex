\section{Introduction}

Autonomous vehicles are expected to have a great impact on how the traffic situation
in our cities will look like in the not so far future. During the last years, autonomous
vehicle technology took great leaps forward, examples being Google's self-driving cars
\citep{Google2016} or Tesla's latest software updates on autonomous parking functionality \citep{Tesla2016}.
Furthermore, renowned car manufacturers such as Audi recently joined the competition.
Ambitious tests in real-world conditions are fostered all over the world, examples being Volvo's
DriveMe project with 100 autonomously driving cars on the highways of Gothenburg \citep{Gothenburg2016},
the LUTZ pathfinder project, establishing the use of three autonomous transport pods
in the city center of Milton Keynes \citep{MiltonKeynes2016} and a case study of
autonomous taxis in Singapore \citep{LTA2016}.

Especially big cities like Singapore could gain a lot from autonomous car technology,
be it from privately owned cars to publicly owned services. A floating fleet of
autonomous vehicles (AVs) could have a drastic impact on land usage, reducing the area of parking
space needed in the urban environment. At the same time, it has the potential improve road safety, access to transportation,
congestion, and emissions \citep{Kheong2014}. In fact, it is predicted
that an AV fleet size of one-third the size of the number of private cars could
serve the associated demand that is generated today \citep{Spieser2014}. From a user perspective, shared autonomous taxis could solve the classic ``last mile problem'' where AVs could
bridge the transport gap from the closest public transport facility to the homes
of the people \citep{Litman2014}. As soon as a particular supply of AVs is reached, prices are predicted
to become highly competitive to if not even cheaper than private car ownership \citep{Chen16}.

On the other side, with the adaptation of autonomous transport, fundamental moral
problems \citep{Hevelke2015a} need to be discussed, and related questions in liability and
ownership need to be answered \citep{Anderson2014a}. In general, predicting how and when autonomous
vehicles will be on the roads is a highly complex problem, and most of the proclaimed
benefits can easily be defeated due to a lack of reliable data.

To give an example, one of these benefits is having less congestion \citep{ITF2014}. However,
it is claimed that to make a passenger comfortable in a self-driving car,
significantly smaller accelerations and decelerations are allowed, and thus the
overall movements on the roads will be slowed down \citep{LeVine2015}. Therefore, having only assumptions
on how people would experience shared AV services, replacing a huge percentage of
contemporary cars with autonomous ones could in fact increase overall congestion.

Another example is the overall travel demand, which might increase with the introduction
of autonomous vehicles as a consequence of people seeing it as more convenient and
cost-efficient to other modes. This, in turn, could lead to the aversive effect
of increased congestion \citep{Litman2014}.

In consequence, while the technological development in autonomous driving already
came a long way, there is still much demand for research on an upper level, looking
at the overall economic and societal picture \citep{Silberg2013, SchoettleBrandon2014}. A valuable tool for doing this is the
use of traffic simulation \citep{Fagnant2014, ITF2014, Zachariah13}. Numerous efforts have been
made in predicting the usefulness of AVs for a city regarding AV supply, i.e.
answering the question how many shared AVs would be needed to serve an
existing demand fraction \citep{Bosch2015, Fagnant2015Austin}. On the contrary, fewer results are available on how the new travel
mode would interact with existing travel options and how customer preferences
influence the mode choice.

The agent- and activity-based traffic simulation framework MATSim \citep{Horni2015} allows for such
research \citep{Boesch2015} by taking into account assumed customer preferences and letting people
interact with a newly added means of transportation.
The framework has been successfully used in a range of studies from autonomous taxi services
in Berlin and Barcelona \citep{Bischoff2016} up to the simulation of the whole transport network of
Singapore \citep{Erath2014}. While the framework allows for a quite precise prediction of traffic
flows for scenarios involving car traffic and public transport, it yet does not
allow the simulation of dynamically acting autonomous vehicles embedded in the
overall traffic situation.

Therefore, the purpose of this thesis will be to implement means of simulating autonomous
taxis within the MATSim framework. Subsequently, measurements on how people would
switch to the new transport mode given certain supply levels, acceptance levels, and
pricing schemes will be made. A major challenge will be to account for an acceptable simulation
speed while keeping the dynamic detail, which is needed in order to simulate
intelligently acting AV taxi fleets. At the same time the functionality should be kept
as versatile and extendable as possible in order to make it possible to gain research
results on a multitude of factors, which influence the adaptation of autonomous
vehicles. Based on these criteria, a new framework for simulating dynamic agents
has been developed (\cref{sec:dynagent}).

In the scope of this thesis, a basic model of autonomous vehicles, which are transporting one passenger, is developed. Many
interesting aspects such as shared AVs, AVs for the purpose of feeding
public transport facilities or simulating a refuel/recharging infrastructure can
be subject of future research. Since the model relies on how likely
people are to use autonomous technology, the model at the moment will be mainly
pointed towards finding qualitative results on the interplay between factors that
define the traffic situation. However, customer preferences, actual experiences of AV technology and pricing information will
become available over the next years.

Given those data sets, the simulation developed in the thesis has the potential
to act as a valuable tool in transport planning and urban development. It will be useful
in planning for the implementation of the needed infrastructure, the development of
pinpointed transport solutions and a restructuring of the present traffic network
to account for the adaptation of autonomous vehicles.

The thesis at hand is structured in four main parts: \Cref{sec:matsim} will introduce how
traffic simulation is performed in the MATSim framework and highlight the aspects
that are important to know for the implementation of autonomous vehicles. \Cref{sec:sioux},
as another prerequisite for setting up a meaningful simulation, covers the adaptation
of the readily available MATSim scenario of the city of Sioux Falls. In \cref{sec:dynagent,sec:avmodel}, the technical implementation
of the AV simulation will be explained in detail on two abstraction layers, first
introducing a new framework for simulating dynamically acting agents in MATSim
and then creating a model of autonomous taxis based on that. Then, in \cref{sec:results}
the model will be tested with a variety of parameter configurations, giving
insights on the general working of the model and predictions of AV usage in the
artificial Sioux Falls test scenario. Finally,
\cref{sec:conclusion} provides an overview about the qualitative results, while
\cref{sec:outlook} will give an outlook on
a multitude of possible extensions of the model and further research questions that
can be answered using the model at hand.
