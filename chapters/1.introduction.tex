\section{Introduction}

\begin{itemize}

\item General introduction, MATSim is suitable for simulating autonomous vehicles \citep{Boesch2015} etc ...

\item Sovles last-mile problem \citep{Litman2014}

\item Interesting previous studies \citep{Fagnant2014, ITF2014}

\item Progress \citep{Silberg2013, SchoettleBrandon2014}

\item Singapore \citep{Kheong2014}

\end{itemize}

\todo{Introduction}

Autonomous vehicles are expected to have great impact on how the traffic situation
on the road will look like in the not so far future. During the last years autonomous
vehicle technology took great leaps forward, examples being Google's self-driving cars
\todo{cite} or Tesla's latest software updated on semi-autonomous driving on highways \todo{cite}.
Furthermore, renowned car manufacturers such as Audi recently joined the competition
and tests in real-world conditions are fostered all over the world as in Volvo's
DriveMe project on the highways of the city of Gothenburg, autonomous transport pods
in the city of Milton Keynes \todo{cite, cite}.

Especially big cities like Singapore could gain a lot from autonomous car technology
be it from privately owned cars to publicly owned services. A floating fleet of
AVs could have a drastical impact on land usage, reducing the place of parking
space needed in the urban environment, and improve road safety, access to transportation,
congestion and emissions at the same time. \todo{cite}
From from a user perspective, shared autonomous taxis could solve the classic ``last mile problem'' where AVs could
bridge the transport gap from the closest public transport facility to the homes
of the people. As soon as a certain supply of AVs is reached, prices are predicted
to be lower and more evenly distributed compared to current pricing schemes for
public services \todo{cite}.

On the other side, with the adaptation of autonomous transport, fundamental moral
problems \todo{cite} need to be solved and related questions in liability and
ownership needs to be answered. In general, predicting how and when autonomous
vehicles will be on the roads is a highly complex problem and most of the proclaimed
benefits can easily be defeated due to a lack of reliable data.

To give an example, one of these benefits is having less congestion \todo{cite}. However,
it is claimed that in order to make a passenger comfortable in a self-driving car,
significantly smaller accelerations and decellerations are allowed and thus the
overall movements on the roads will be slowed down \todo{cite}. Having only assumptions
on how many people would take part in shared AV services, replacing a huge share of
contemporary cars with autonomous ones could actually increase overall congestion.

In consequence, while the technological development in autonomous driving already
came a long way, there is still a lot of demand for research on an upper level, looking
at the overall economic and societal picture. While numerous efforts have been made
in predicting the usefulness of AVs for a city in terms of AV supply by, for
instance, answering the question how many shared AVs would be needed to serve the
existing demand \todo{cite}, less results are available on how the new travel
mode would interact with existing travel options and how customer preferences
influence the mode choice.

The agent- and activity-based traffic simulation framework MATSim allows for such
research by taking into account assumed customer preferences and letting the people
interact with a newly added means of transportation. Consequently, this allows one
to predict the shares of autonomous vehicle rides for certain levels of acceptance
for the technology, different pricing schemes and different levels of availability.
The framework has been successfully used in a range of studies from taxi services
in Berlin and Barcelona up to the simulation of the whole transport network of
Singapore. While the framework allows for a quite precise prediction of traffic
flows for scenarios involving car traffic and public transport, it yet does not
allow the simulation of dynamically acting autonomous vehicles.

Therefore the purpose of this thesis will be to implement means of simulating autonomous
taxis within the MATSim framework. Major challenges will be to account for an
accptable simulation speed while keeping the dynamic detail, which is needed in
order to simulate intelligently acting taxi fleets and to keep the added functionality
as versatile and extendable as possible in order to make it possible to gain
research results on a multitude of factors that influence the adaptation of
autonomous vehicles.

This, however, also means that in the scope of this thesis the basic model of
autonomous vehicles, which are transporting one passengers, is developed, while
many interesting aspects such as shared AVs, AVs for the purpose of feeding
public transport facilities or simulating a refuel/recharging infrastructure can
be subject of future research. Also, since the model relies on data on how likely
people are to use autonomous technology, only vague predictions can be made at
the moment, based on current predictions. However, more and more data on customer
preferences, actual experiences of AV technology and pricing information will
become available over the the next years, leading the model to give a more accurate
and reliable look into the future.

Given those data sets, the simulation developed in the thesis has the potential
to act as a valuable tool in transport planning and urban development. It will give
great aid in the implementation of the needed infrastructure, the development of
pinpointed transport solutions and a restructuring of the present traffic network
to account for the adaptation of autonomous vehicles.

The thesis at hand is structured in four main parts: Chapter 2 will introduce how
traffic simulation is performed in the MATSim framework and highlight the aspects
that are important to know for the implementation of autonomous vehicles. Chapter 3,
as another prerequisite for setting up a meaningful simulation, covers the adaptation
of a readily available MATSim traffic scenario to the increased network resolution,
which is needed here. In Chapters 4 and 5 the technical implementation
of the AV simulation will be explained in detail on two abstraction layers, first
introducing a new framework for simulating dynamically acting agents in MATSim
and then creating a model of autonomous taxis based on that. Then, in Chapter 6
the model will be tested with a variety of parameters configurations, giving
insights on the general working of the model and predictions of AV usage in the
artificial Sioux Falls test scenario. Finally, Chapter 7 will give an outlook on
a multitude of possible extensions of the model and further research questions that
can be answered using the model at hand.
